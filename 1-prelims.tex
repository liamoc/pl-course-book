\documentclass{book}
\usepackage{amsmath}
\usepackage{amsthm}
\usepackage{xcolor}
\usepackage{tikz}
\usetikzlibrary{calc}

\usepackage{lipsum}

\usepackage{pxfonts}

\usepackage{relsize}
\usepackage{textgreek}
\usepackage{standalone}

\tikzset{fontscale/.style = {font=\relsize{#1}}}


\usepackage{haskell}
\usepackage{mathpartir}
\usepackage{grammar}

\usepackage{ifthen}


\newcommand{\mapstos}[0]{\stackrel{\star}{\mapsto}}
\newcommand{\mapstoc}[0]{\stackrel{!}{\mapsto}}
\newcommand{\rel}[2]{{#1}\;\longleftrightarrow\;{#2}}
\newcommand{\subst}[3]{\{{#2}/{#1}\}#3}
\newcommand{\te}[1]{{\textnormal{\texttt{#1}}}}
\newcommand{\tearg}[2]{{\texttt{#1(}#2{\texttt{)}}}}
\newcommand{\ok}[1]{{#1}\;\small{\textbf{ok}}}

\usepackage[lastexercise,answerdelayed]{exercise}


\newcommand{\pred}[2]{{#1}\;\textsf{\textbf{#2}}}
\newcommand{\predand}[2]{{#1}\;\textsf{\textbf{#2}}\quad}
\newcommand{\gnm}[1]{\text{\gntermfont{#1}}}

\begin{document}
\chapter{Natural Deduction and Inductive Definitions}
\section{Introduction}

As we are going to discuss and reason about properties of various programming
languages and language features, we need a formal meta-language which allows
us to make statements about these properties. We need to specify the grammar
of a language, the static semantics (often in form of typing and scoping
rules) and dynamic semantics. Fortunately, it turns out that a single
formalism, inductive definitions built on inference rules, is sufficient.



\section{Judgements and Inference Rules}
A \emph{judgement} is simply a statement that a certain property holds for a
specific object,\footnote{More generally, a relationship between a number of
  objects holds. For now, we just look at statements about a single object.} or alternatively 
\begin{itemize}
\item \<3 + 4 * 5\> is a valid arithmetic expression
\item the string \<"aba"\> is a palindrome
\item \<0.43423\> is a floating point value
\end{itemize}
Judgements are not unlike  predicates you might know from Predicate Logic. We
write \[ a\; S \]
for a judgement of the form \textit{The property S holds for object a}. In
predicate logic, this is usually written differently, as
\textit{S(a)}. However, we will see later that for our purpose, it is much
more convenient to write it in the above given post-fix
notation.  Alternatively, we can interpret $S$ as a set of objects with a
certain property, and read the judgement $a S$ as: $a$ is an element of the set
$S$. Some examples of judgements and how to read them are:
\begin{itemize}

  \item[-] 5 \textit{even}
    \begin{itemize}   
    \item 5 is even, or 
    \item 5 is an element of the set of even numbers
    \end{itemize}
    
  \item[-] \<3 + 4 * 5\>  \textit{expr}
    \begin{itemize}   
    \item \<3 + 4 * 5\> is a syntactically correct expression, or
    \item \<3 + 4 * 5\> an element of the set containing all syntactically correct expressions
    \end{itemize}

 \item[-] \<0.43423\>  \textit{float}
    \begin{itemize}  
      \item \<0.43423\> is a floating point value
      \item \<0.43423\> is an element of the set of floating points values
    \end{itemize}
  \end{itemize}
Judgements by themselves would be boring and fairly useless. Most
interesting sets have an infinite number of elements, and to define such a set
it would obviously be impossible to explicitly list them all using simple
judgements. Luckily, the sets we are interested in are also no random
collections of objects, but the sets and properties can be systematically
defined by so-called \emph{inference rules}.

Inference rules allow us to combine judgements to obtain new judgements and
have the following general form:
\begin{quote}
   If  judgements \<J_1\>, and \<J_2\>, and \<\ldots\> and $J_n$ are
   \emph{inferable}, then 
   the judgement \<J\> is   \emph{inferable}
 \end{quote}

and are usually given in the following standard form:
\[
\inferrule{J_1\;\; J_2\;\; \ldots\;\; J_n}{J}
\]
where the judgements $J_1$ to $J_n$ are called \emph{premises}, and $J$ is
called a \emph{conclusion}. An inference rule does not have to have premises,
it can consist of a single conclusion. Such inference rules are called
\emph{axioms}. But let us have a look at a concrete example now.

We start by defining some simple properties over the set of  natural numbers
(\textit{Nat}). For  simplicity reasons, we represent them as \<0, s(0), s(s(0)),
s(s(s(0)))\>, and so on ($s$ here stands for \textit{successor}). So, first of all, how can we define \textit{Nat} itself
using  inference rules? Listing all the element of \textit{Nat} would be equivalent
to including an axiom for each number:
\begin{haskell}
\inferrule{}{\pred{0}{Nat}}\\ \phantom{a}\\
\inferrule{}{\pred{s(0)}{Nat}}\\ \phantom{a}\\
\inferrule{}{\pred{s(s(0))}{Nat}}\\ \phantom{a}\\
\vdots
\end{haskell}
Apart from the first rule, all the rules have the form 
\begin{haskell}
  \inferrule{}{\pred{s(x)}{Nat}}
\end{haskell}
where $\pred{x}{Nat}$ has been established by the previously listed axiom. In
words, we have 
\begin{enumerate}
  \item $0$ is in \textit{Nat}, and
  \item for all $x$, if $x$ is in \textit{Nat}, then  $s(x)$  in \textit{Nat} 
\end{enumerate}
which can be translated directly into the following two inference rules:
\begin{haskell}
\inferrule{}{\pred{0}{Nat}}\\ \phantom{a}\\
\inferrule{\pred{x}{Nat}}{\pred{s(x)}{Nat}}
\end{haskell}
where $x$ can be instantiated to any In the same way, we can define the sets \<Even\> and \<Odd\>: 

\begin{minipage}{0.5\textwidth}
\begin{haskell}
\inferrule{}{\pred{0}{Even}}\\ \phantom{a}\\
\inferrule{\pred{x}{Even}}{\pred{s(s(x))}{Even}}
\end{haskell}
\end{minipage}
\begin{minipage}{0.5\textwidth}
\begin{haskell}
\inferrule{}{\pred{s(0)}{Odd}}\\ \phantom{a}\\
\inferrule{\pred{x}{Odd}}{\pred{s(s(x))}{Odd}}
\end{haskell}
\end{minipage}

Rules do work in two ways: we can use them to define a property, but we can
also use them to show that a judgement is valid. How does it work with
inference rules? Assume we want to show that some judgement $J$ is valid. We
have to look for a rule which has $J$ as a conclusion. If this rule is an
axiom, we are already done. If not, we have to show that all of its premises
are valid by recursively applying the same strategy to all of them. For 
example, we can show that \<\pred{s(s(0))}{Even}\>, since 
\[\inferrule{\pred{0}{Even}}{\pred{s(s(0))}{Even}}\]
and
\[\inferrule{}{\pred{0}{Even}}\] 
As the last rule is an axiom, there are no premises left to prove, and we are
done. 

Similarly, we can show that
\<\pred{0+1+1+1+1}{Even}\>. An alternative and often quite convenient way to
write inference proofs is to stack the rules we apply together and draw a
``proof tree'' --- in our example, more a proof stack, since each rule has
just a single premise. 
\[
\inferrule{
  \inferrule{
    \inferrule{
      \inferrule{}{
      \pred{s(0)}{Odd}}}{
      \pred{s(s(0))}{Odd}}
  }{\pred{s(s(s(0)))}{Odd}}
}{
  \pred{s(s(s(s(s(0)))))}{Odd}
}
  \]



Note that inference works on a purely syntactic
basis. Given the rules above, we are not able to prove \<\pred{2}{Even}\>,
even though we can show that \<\pred{s(s(0))}{Even}\>, and we know that
\<s(s(0))\> is equal to \<2\>, we cannot apply that knowledge, since we have no
rule which tells us it is ok to do so. We just mechanically 
manipulate terms according to the given rules.


Let us look at a slightly more interesting example: we want to define the
language\< M\> which contains all expressions of properly matched parenthesis
(and no other characters):\footnote{$\epsilon$ represents the empty string}
\[M = \{\epsilon, \mathtt{()}, \mathtt{()()}, \mathtt{()()()}, \ldots, \mathtt{(())}, \mathtt{((()))}, \ldots, \mathtt{()(())},\mathtt{()()(())}, \ldots \}\] 

Again, let us start by describing the rules in
(semi-)natural language. There are basically two ways to ``legally'' combine
parenthesis: we can either nest them, or concatenate them:
\begin{enumerate}
\item  The empty string (denoted  by $\epsilon$) is in \<M\>
\item  If \<s_1\> and \<s_2\> are in \<M\>, then \<s_1s_2\> is in \<M\>
\item   If \<s\> is in \<M\>, then \<\mathtt{(}s\mathtt{)}\> is in \<M\>
\end{enumerate}
Again, these rules can be directly translated into inference rules:

\begin{eqnarray*}
(1)&\qquad\phantom{a} &  \Large{\inferrule{\phantom{asdasd}}{\pred{\epsilon}{M}}}\\ \phantom{a}\\
(2)& &  \Large{\inferrule{\predand{s_1}{M}\pred{s_2}{M}}{\pred{s_1s_2}{M}}}\\ \phantom{a}\\
 (3)&& \Large{\inferrule{\pred{s}{M}}{\pred{\mathtt{(}s\mathtt{)}}{\;M}}}
\end{eqnarray*}

How can we show that  \<\mathtt{()(())}\> is in \<M\>? As we did previously,
we check if there is a rule whose conclusion matches the judgement we want to
infer. If we apply Rule~(2), we have to show that both $()$ and $(())$
are in $M$. Since $()=(\epsilon)$, we can apply Rule~(3), and only have to
show that $\epsilon$ is in $M$ (Rule~(1)). By applying Rule~(3) in the same
way, we can show that $(())$ is in $M$ as well, and we are done: 
%
The problem is, however, not as straight forward as it seems, because instead
of applying Rule~(2), we could as well apply Rule~(3). It has
only a single premise, so we only have to prove that $)(()$ is in $M$. The
only rule that's applicable now is Rule~(2), and we could apply it in
several different ways resulting in different premises:
\[\inferrule{\predand{)}{M}\pred{\mathtt{(()}}{M}}{\pred{\mathtt{)(()}}{M}}\]
or
\[\inferrule{\predand{\mathtt{)(}}{M}\pred{\mathtt{()}}{M}}{\pred{\mathtt{)(()}}{M}}\]
or
\[\inferrule{\predand{\mathtt{)((}}{M}\pred{\mathtt{)}}{M}}{\pred{\mathtt{)(()}}{M}}\]
In the first application, we end up with the premise: $\mathtt{)}$ is in $M$, but there
is no rule which we can apply to get rid of it. This is not that
surprising, since $M$ should only contain expressions of properly matched
parenthesis, and $\mathtt{)}$, as well as $\mathtt{)(()}$ are not properly matched. So, by
choosing the wrong rule, or applying the right rule in a wrong way -- for example,
splitting $\mathtt{()()}$ up into $\mathtt{()(}$ and $\mathtt{)}$ --
it is easily possible to end up with premises which
are not actually valid and reach a dead end. In our example, this was not hard
to see. It can be extremely difficult to decide which rule to apply
and how, without some background knowledge about the objects and properties, as
there might be an infinite number of possibilities. This is one reason why it
is not possible to write a program which automatically infers judgements, and
which guarantees to find such a derivation if it exists. It is, however,
possible to write semi-automatic theorem provers, which come up with proofs in
cases where it is fairly standard, and rely on user input otherwise. 


\subsection{Derivable and  Admissible Rules}
What would happen if we added the following rule:
\[
(4)\qquad\inferrule{\pred{s}{M}}{\pred{\mathtt{((}s\mathtt{))}}{M}}
\]
Does this change the set $M$ in any way, that is, is there a string $s$ for which
we can infer $\pred{s}{M}$ if we use $4$, but not otherwise? This doesn't
seem very likely: the rule just says that, if a string $s$ is in $M$ it is ok
to add two pairs of matching parenthesis. Since we already had a rule which
allows us to add one pair of parenthesis, we can apply this rules twice and
achieve the same effect: 
\[
\qquad\inferrule{\inferrule{\pred{s}{M}}{\pred{\mathtt{(}s\mathtt{)}}{M}}}{\pred{\mathtt{((}s\mathtt{))}}{M}}
\]
This means that Rule $(4)$ is \emph{derivable} from the existing
rules. 

In all the previous rules, the strings in the premises were simpler than the
string in the conclusion. The following rule if different in this respect: 
\[
(5)\qquad\inferrule{\pred{\mathtt{()}s}{M}}{\pred{s}{M}}
\]
Interestingly, although Rule~$(4)$  does not introduce any new strings to $M$, the rule
is also not derivable from any of the existing rules. Such  a rule is called
\emph{admissible}.


\section{Induction} 

Natural deduction by itself is sometimes not powerful enough. For example,
although we can see that the Rule $(5)$ is valid for every string $s$
in $M$, we cannot show this by simply combining the existing rules. We will
therefore introduce another proof technique here, called \emph{induction}. You
will probably know induction over natural numbers and structural induction
from mathematics and previous courses. Both are special cases of a more
general induction principle called \emph{rule induction}. 


A set of inference rules $R$ defining a set $A$ is called an \emph{inductive
  definition} of $A$, if $\pred{s}{A}$ holds if and only if $\pred{s}{A}$ is
derivable using $R$. Not all sets can be defined using inductive
definitions. For example, while natural numbers are one of the standard
examples for such sets, floating point numbers cannot be defined in such a
way. 



\subsection{Rule Induction}
Let us go back to our previous example set $M$ of properly matched
parenthesis. The rules $1$ to $3$ provide an inductive definition of $M$: 
\begin{eqnarray*}
(1)&\qquad\phantom{a} &  \Large{\inferrule{\phantom{asdasd}}{\pred{\epsilon}{M}}}\\ \phantom{a}\\
(2)& &  \Large{\inferrule{\predand{s_1}{M}\pred{s_2}{M}}{\pred{s_1s_2}{M}}}\\ \phantom{a}\\
 (3)&& \Large{\inferrule{\pred{s}{M}}{\pred{\mathtt{(}s\mathtt{)}}{\;M}}}
\end{eqnarray*}
Now, let us assume we want to prove some property $P$ of the strings in $M$,
that is: show that if $\pred{s}{M}$ then $\pred{s}{P}$. Since we know that
there is a derivation for each $\pred{s}{M}$, we  only need to show
that:
\begin{itemize}
  \item $\pred{\epsilon}{P}$
  \item if  $\pred{s_1}{P}$ and $\pred{s_2}{P}$, then $\pred{s_1 s_2}{P}$
  \item if  $\pred{s}{P}$, then $\pred{(s)}{P}$
\end{itemize}
Which in essence correspond to the original rules only with $M$ replaced by $P$. For
example, if we want to show that all $s$ in $M$ have the same number of opening and
closing brackets, we need to prove the following statements (where
\textit{open} denotes the number of opening, \textit{close} those of closing brackets):
\begin{enumerate}
\item   \<open({\epsilon}) = close (\epsilon)\>

  \textbf{Proof:}   \<open({\epsilon}) = 0 = close (\epsilon)\>

\item if \<open({s_1}) = close (s_1)\> and \<open({s_2}) = close (s_2)\> then 
  \<open({s_1 s_2}) = close (s_1 s_2)\>

  For the proof, we assume that the statements corresponding to the judgements
  in the premises of the rules hold. These assumptions are called the
  \emph{induction hypothesis}. 

  \begin{itemize}
    \item Induction Hypothesis 1:  \<open({s_1}) = close (s_1)\>

    \item Induction Hypothesis 2:   \<open({s_2}) = close (s_2)\>
  \end{itemize}
  
  \textbf{Proof:} \<open({s_1 s_2}) = open(s_1) + open (s_2) =close(s_1) +
  close (s_2) = close (s_1 + s_2)\>
  
\item if \<open({s}) = close (s)\> then \<open(\mathtt{(}s\mathtt{)})\> 
  \begin{itemize}
  \item Induction Hypothesis 1:  \<open({s}) = close (s)\>
  \end{itemize}
    
  \textbf{Proof:} 
\begin{displaymath}
\begin{array}{lcl}
	&& open(\mathtt{(}s\mathtt{)}) \\
	&=& \text{\{property of \textit{open}\}}\\
   && open(\mathtt{(}) + open(s) +open(\mathtt{)})\\
	&=& \text{\{property of \textit{open}\}}\\
&& 1  + open (s) + 0  \\
		&=& \text{\{Induction Hypothesis,Arithmetic\}}\\
		 && 1  + close (s) + 0 \\
		&=& \text{\{property of \textit{close}\}}\\
	   && close(\mathtt{(}) + close(s) + close(\mathtt{)})\\
				&=& \text{\{property of \textit{close\}}}\\
					&& close(\mathtt{(}s\mathtt{)}) 
\end{array}
\end{displaymath}

\end{enumerate}

In the proof, we used the rules of arithmetic, and that properties of \<close\> and \<open\>, such as \<open(\mathtt{(}) = close (\mathtt{)}) = 1\>, and \<open(\mathtt{)})= close(\mathtt{(}) = 0\>. In a fully formal proof, we would also need a formal definition of these two functions.


\subsection{Ambiguity}
The definition of $M$, although correct, has a undesirable property: for
any string in $M$, we do not have just one derivation, but an
infinite number of possible derivations, since any string $s$ can be split
into  $\epsilon$ and $s$ by applying Rule~(2), and then Rule~(1) to get rid of
 $\epsilon$. However, this derivation step is completely unnecessary. 

 Fortunately, if we take a more structured view on the elements of this
 language, we can come up with an alternative set of rules, where we have
 exactly one derivation for each string in the set. We can interpret each
 string as a possibly empty list ($L$) of non-empty parenthesis expressions
 ($N$) according to the following inference rules: 
\begin{eqnarray*}
(1) & \inferrule{}{\pred{\epsilon}{L}}\\
(2) & \inferrule{\predand{s_1}{N}\pred{s_2}{L}}{\pred{\epsilon}{L}}\\
(2) & \inferrule{\pred{s}{L}}{\pred{\mathtt{(}s\mathtt{)}}{N}}\\
\end{eqnarray*}
The interesting point here is that $L$ and $N$ are defined in terms of each
other: we have a mutually recursive definition. 

Let us look at one more example of an ambiguous definition: simple
arithmetic expressions, given here both in EBNF form and as inductive
definition using inference rules:

The EBNF
  \begin{grammar}
     \grule{Expr}{%
       \gnterm{int}\ \galt\ (\gnterm{Expr}) 
       \galt\ {\gnterm{Expr} + \gnterm{Expr}\ 
       \galt\ {\gnterm{Expr} * \gnterm{Expr}}}}
   \end{grammar}
 describes the same language as the following set of inference rules ($\gnm{int}$  represents an integer constant):
  \begin{gather*}
\inferrule{\pred{e}{\gnm{Expr}}}{\pred{(e)}{\gnm{Expr}}}\qquad
\inferrule{\predand{e_1}{\gnm{Expr}} \pred{e_2}{\gnm{Expr}}}{\pred{e_1\;\texttt{+}\;\;e_2}{\gnm{Expr}}}
\qquad
\inferrule{\predand{e_1}{\gnm{Expr}}\pred{e_2}{\gnm{Expr}}}{\pred{e_1\;\texttt{*}\;\;e_2}{\gnm{Expr}}}
  \end{gather*}
Although in this case, there are not an infinite number of possible
derivations for each expression, every expression which contains more than a
single arithmetic operation still can be derived in more than one way:
\begin{gather*}
    \inferrule{\inferrule{\inferrule{}{1\; \gnm{Expr}}\quad
        \inferrule{}{2\; \gnm{Expr}}}{1 + 2\; \gnm{Expr}}\;\;
      \inferrule{}{3\;\gnm{Expr}}}{%
        1+2*3\;\gnm{Expr}}
  \end{gather*}
\begin{gather*}
    \inferrule{
      \inferrule{}{1\;\gnm{Expr}}\;\;
\inferrule{\inferrule{}{2\; \gnm{Expr}}\quad
        \inferrule{}{3\; \gnm{Expr}}}{2 * 3\; \gnm{Expr}}
    }{%
        1+2*3\;\gnm{Expr}}
  \end{gather*}

  Although both derivations are correct with respect to the rules given, the
  second derivation is more appropriate for an arithmetic expression, as it
  decomposes the expression first into two summands. We give an alternative
  definition, which takes precedence and associativity  of the operators into
  account:
  
  \begin{gather*}
\inferrule{\predand{e_1}{\gnm{SExpr}}\pred{e_2}{\gnm{PExpr}}}{\pred{e_1\;\;\texttt{+}\;e_2}{\gnm{SExpr}}}
\qquad
\inferrule{\pred{e}{\gnm{PExpr}}}{\pred{e}{\gnm{SExpr}}}
\end{gather*}

\begin{gather*}
\inferrule{\predand{e_1}{\gnm{PExpr}}\pred{e_2}{\gnm{FExpr}}}{\pred{e_1\;\texttt{*}\;\;e_2}{\gnm{PExpr}}}
\qquad
\inferrule{e\; FExpr}{e\;PExpr}
  \end{gather*}


\begin{gather*}
\inferrule{e\; SExpr}{\texttt{(}e\texttt{)}\;FExpr}\qquad
\inferrule{}{int\;FExpr}
  \end{gather*}
The unambiguous grammar is, again, much more complicated than the original
grammar, even for such a simple language. This is not surprising, as it
contains additional structural information. For programming languages, ambiguous
grammars are problematic, as they may allow different interpretations of
programs, and are therefore usually avoided. 

\subsection{Simultaneous Induction}
How can we apply the principle of rule induction to mutually recursive
definitions like those of $L$ and \textit{SExpr}? In most cases, we have to
generalise the proof goal. For example, if we want to prove a property $P$ for all
$e$ in \textit{SExpr}, that is \<\pred{e}{SExpr}\> implies \<\pred{e}{PExpr}\>.
By the principle of rule induction we have to show that 
\begin{itemize}
\item under the assumption that
\begin{itemize}
 \item[-] \<e = e_1 + e_2\>
 \item[-] \<\pred{e_1}{SExpr}\>
 \item[-] \<\pred{e_2}{PExpr}\>
 \end{itemize}
 and the Induction Hypothesis 
\begin{itemize}
  \item[-]  \<\pred{e_1}{P}\>
  \end{itemize}
  show that   \<\pred{e_1+e_2}{P}\>
\item under the assumption that
  \begin{itemize}
  \item[-] \<\pred{e}{PExpr}\>
  \end{itemize}
  show that \<\pred{e}{P}\>
\end{itemize} 

The problem is that, since we only try to show something about
\<SExpr\>, the induction hypothesis does not say anything about \<e_2\> (we
only know that \<\pred{e_2}{PExpr}\>, and nothing at all about $e$ in for the
second rule. In most cases, this is not enough to prove anything. The
solution is often to try and prove a more general statement instead which
leads to stronger induction hypothesis. If we try to show, for instance, that 
\<\pred{e}{SExpr}\> or \<\pred{e}{PExpr}\> or \<\pred{e}{FExpr}\> implies \<\pred{e}{P}\>,
we have more cases to cover on one hand (one for each inference rule which has
\textit{PExpr}, \textit{SExpr} or \textit{FExpr} in the conclusion), on the other hand the induction
hypothesis cover all the premises in the rules:
\begin{itemize}
\item under the assumption that
\begin{itemize}
 \item[-] \<e = e_1 + e_2\>
 \item[-] \<\pred{e_1}{SExpr}\>
 \item[-] \<\pred{e_2}{PExpr}\>
 \end{itemize}
 and the Induction Hypothesis 
\begin{itemize}
  \item[-]  \<\pred{e_1}{P}\>
  \item[-]  \<\pred{e_2}{P}\>
  \end{itemize}
  show that   \<\pred{e_1+e_2}{P}\>
\item under the assumption that
  \begin{itemize}
  \item[-] \<\pred{e}{PExpr}\>
  \end{itemize}
 and the Induction Hypothesis 
 \begin{itemize}
 \item[-]  \<\pred{e}{P}\>
 \end{itemize}
 show that \<\pred{e}{P}\> (trivial)
\item under the assumption that
\begin{itemize}
 \item[-] \<e = e_1 * e_2\>
 \item[-] \<\pred{e_1}{SExpr}\>
 \item[-] \<\pred{e_2}{PExpr}\>
 \end{itemize}
 and the Induction Hypothesis 
\begin{itemize}
  \item[-]  \<\pred{e_1}{P}\>
  \item[-]  \<\pred{e_2}{P}\>
  \end{itemize}
  show that   \<\pred{e_1*e_2}{P}\>
  
\item \ldots
\end{itemize} 


\section{Judgements and Relations}
So far, we defined a judgement to be a statement about a property of an
object. We can generalise this definition to relation between a number of
objects. Consider the following inductive definition of the  relation ``a divides b'' on natural
numbers. For convenience reasons, we choose an infix notation here:
  \begin{gather*}
\inferrule{}{0\;div\; n}\\\phantom{a}\\
\inferrule{n\; div\; m}{n\; div\; (m+n)}
\end{gather*}
As before, we can also view this as an inductive definition of a set. In this
case, a set of pairs, where $(a,b)$ in \textit{div} if and only if $a$ divides
$b$. Rule inductions works exactly in the same way as in the unary case.

\section{Boolean Expression Example}		
As another example, consider boolean expressions. For simplicity, we only include three operators for now: $\wedge$, $\vee$, and $\neg$, the constants \textit{True} and \textit{False}, and parentheses. Our first attempt at defining a set of inference rules to characterise boolean expressions might look as follows:

  \begin{gather*}
	\inferrule{\phantom{bla}}{\pred{\bf{True}}{\gnm{BExpr}}}\qquad \inferrule{\phantom{bla}}{\pred{\bf{False}}{\gnm{BExpr}}}\qquad \inferrule{\pred{e}{\gnm{BExpr}}}{\neg\pred{e}{\gnm{BExpr}}}\\ \\
\inferrule{\pred{e}{\gnm{BExpr}}}{\pred{(e)}{\gnm{BExpr}}}\qquad
\inferrule{\predand{e_1}{\gnm{BExpr}} \pred{e_2}{\gnm{BExpr}}}{\pred{e_1 \wedge e_2}{\gnm{BExpr}}}
\qquad
\inferrule{\predand{e_1}{\gnm{BExpr}}\pred{e_2}{\gnm{BExpr}}}{\pred{e_1\vee e_2}{\gnm{BExpr}}}
  \end{gather*}

Unfortunately, with this set of rules, we have the same problem we had with our rules for arithmetic expressions. Even though they inductively define the set of boolean expressions, they are ambiguous and do not reflect  associativity and precedence of the operators. So, we need to come up with an alternative definition. The operator $\neg$ has the highest precedence, $\vee$ the lowest, and both $\wedge$ and $\vee$ are left associative. The solution is also similar to the solution for arithmetic expressions. First, we need rules to define the subset of boolean expressions  which can be arguments of the operator with the highest precedence, negation. These can only be atomic expressions (constants), any expression in parentheses, or such an expression preceded by negation. Let's call this subset \textit{NbExpr}. We call the boolean expressions we generate with the new rules \textit{Bexpr}.
  \begin{gather*}
	\inferrule{\phantom{bla}}{\pred{\bf{True}}{\gnm{Nbexpr}}}\qquad \inferrule{\phantom{bla}}{\pred{\bf{False}}{\gnm{Nbexpr}}}\qquad \inferrule{\pred{e}{\gnm{Bexpr}}}{\neg\pred{(e)}{\gnm{Nbexpr}}}\\ \\
  \end{gather*}
The rules for the operators $\wedge$ and $\vee$ correspond to those for addition and multiplication. Since the operators are left associative, the expression on the left hand side can only be an expression with stronger cohesion than the one on the right hand side. For the $\wedge$ operator, it has to be a $Nbexpr$.
\begin{gather*}
	\inferrule{\predand{e_1}{\gnm{NbExpr}} \pred{e_2}{\gnm{Abexpr}}}{\pred{e_1\wedge e_2}{\gnm{Abexpr}}}
	\qquad
	\inferrule{\predand{e_1}{\gnm{Abexpr}}\pred{e_2}{\gnm{Bexpr}}}{\pred{e_1\vee e_2}{\gnm{Bexpr}}}
\end{gather*}	
And finally we need rules to express the fact the $Nbexpr \subseteq Abexpr \subseteq Bexpr$
\begin{gather*}
	\inferrule{\pred{e}{\gnm{Nbexpr}}}{\pred{e}{\gnm{Abexpr}}}
	\qquad
	\inferrule{\pred{e}{\gnm{Abexpr}}}{\pred{e}{\gnm{Bexpr}}}
\end{gather*}	


\end{document}