\documentclass{book}
\usepackage{amsmath}
\usepackage{amsthm}
\usepackage{xcolor}
\usepackage{tikz}
\usetikzlibrary{calc}
\tikzset{fontscale/.style = {font=\relsize{#1}}}
\usepackage{lipsum}
\usepackage{hyperref}
\usepackage{pxfonts}
\usepackage{relsize}
\usepackage{textgreek}
\usepackage{standalone}
\usepackage{tabularx}
\usepackage{haskell}
\usepackage{mathpartir}
\usepackage{grammar}
\usepackage{ifthen}
\usepackage[lastexercise,answerdelayed]{exercise}
\usepackage{thmtools}
\declaretheorem[numberwithin=chapter]{theorem}
\declaretheorem[sibling=theorem]{lemma}
\declaretheorem[sibling=theorem]{corollary}
\declaretheorem[numberwithin=chapter,style=definition]{definition}

\newcommand{\todo}[1]{\textbf{TODO}: \emph{#1}}



\newcommand{\mapstos}[0]{\stackrel{\star}{\mapsto}}
\newcommand{\mapstoc}[0]{\stackrel{!}{\mapsto}}
\newcommand{\rel}[2]{{#1}\;\longleftrightarrow\;{#2}}
\newcommand{\subst}[3]{\{{#2}/{#1}\}#3}
\newcommand{\te}[1]{{\textnormal{\texttt{#1}}}}
\newcommand{\tearg}[2]{{\texttt{#1(}#2{\texttt{)}}}}
\newcommand{\ok}[1]{{#1}\;\small{\textbf{ok}}}
\newcommand{\sansbold}[1]{\textsf{\textbf{#1}}}
\definecolor{mygreen}{RGB}{0,149,103}
\newcommand{\meta}[1]{\textcolor{blue}{#1}}
\newcommand{\conc}[1]{\textcolor{mygreen}{#1}}
\newcommand{\pred}[2]{{#1}\;\textsf{\textbf{#2}}}
\newcommand{\predc}[2]{\pred{\conc{#1}}{#2}}
\newcommand{\predm}[2]{\pred{\meta{#1}}{#2}}
\newcommand{\predand}[2]{{#1}\;\textsf{\textbf{#2}}\quad}
\newcommand{\gnm}[1]{\text{\gntermfont{#1}}}

\begin{document}

\section*{Exercises}
\markright{PRELIMINARIES EXERCISES}
\subsection*{1. Strange Loops}
The following system, based on a system called $\sansbold{Miu}$, is perhaps famously mentioned in Douglas Hofstadter's book, \emph{G\"{o}del, Escher, Bach}.
 \newcommand{\conctt}[1]{\conc{\mathtt{#1}}} 
  \begin{displaymath}
    \inferrule{\ }{\predm{\conctt{MI}}{Miu}}{1} \quad \inferrule{\predm{x\conctt{I}}{Miu}}{\predm{x\conctt{IU}}{Miu}}{2} \quad
    \inferrule{\predm{\conctt{M}x}{Miu}}{\predm{\conctt{M}x x}{Miu}}{3} \quad
    \inferrule{\predm{x \conctt{III} y}{Miu}}{\predm{x \conctt{U} y}{Miu}}{4} \quad
        \inferrule{\predm{x \conctt{UU} y}{Miu}}{\predm{x y}{Miu}}{5} 
  \end{displaymath}
  \begin{enumerate}[label=\alph*)]
   \item Is $\pred{\conctt{MUII}}{Miu}$ derivable? If so, show
     the derivation tree. If not, explain why not.
   \item Is $\inferrule{\predm{x\conctt{IU}}{Miu}}{\predm{x\conctt{I}}{Miu}}$ admissible? Is it derivable? Justify your answer.
    \item \textbf{Tricky:} Perhaps famously,
      $\pred{\conctt{MU}}{Miu}$ is not admissible. Prove this
      using rule induction. 

    \emph{Hint}: Try proving something related to the number of $\conctt{I}$s in the string.
  \end{enumerate}
\noindent Here is another language, which we'll call \sansbold{Mi}:
    \begin{displaymath}
      \inferrule{\ }{\predm{\conctt{MI}}{Mi}}{A} \quad \inferrule{\predm{\conctt{M}x}{Mi}}{\predm{\conctt{M}x x}{Mi}}{B} \quad
      \inferrule{\predm{x\conctt{IIIIII}y}{Mi}}{\predm{xy}{Mi}}{C}
    \end{displaymath}
    \begin{enumerate}[label=\alph*)]
     \setcounter{enumi}{3}
     
      \item Prove using rule induction that all strings in $\sansbold{Mi}$ could be expressed as follows, for some $k$ and some $i$, where $2^k - 6i > 0$ (and $\conctt{C}^n$ is shorthand for the character $\conctt{C}$ repeated $n$ times):
      \begin{displaymath}
        \conctt{M}\:\conctt{I}^{2^k - 6i}
      \end{displaymath}
      \item Now prove the opposite claim that, for all $k$ and $i$, assuming $2^k - 6i > 0$:
      \begin{displaymath}
        \pred{\conctt{M}\:\conctt{I}^{2^k - 6i}}{Mi}
      \end{displaymath}
      You should decompose the proof into two lemmas, and prove them separately:
     \begin{enumerate}[label=\roman*.] 
        \item Prove, using induction on the natural number $k$ (i.e cases when $k = 0$ and when $k = k' + 1$), that $\pred{\conctt{M}\:\conctt{I}^{2^k}}{Mi}$
        \item Prove, using induction on the natural number $i$, that $\pred{\conctt{M}\:\conctt{I}^{k}}{Mi}$ implies $\pred{\conctt{M}\:\conctt{I}^{k - 6i}}{Mi}$, assuming $k - 6i > 0$.
     \end{enumerate} 
        As $\pred{\conctt{M\:I}^{2^k}}{Mi}$ for all $k$ is
        established by the first proof, you can conclude from the second proof that $\pred{\conctt{M\:I}^{2^k - 6i}}{Mi}$ for all $k$ and all $i$ where $2^k - 6i > 0$, as required.

      \end{enumerate}
\noindent This completes the proof that the language $\sansbold{Mi}$ is exactly characterised by the formulation $\conctt{M\:I}^{2^k - 6i}$ where $2^k - 6i > 0$. A very neat result!
\begin{enumerate}[label=\alph*)]
     \setcounter{enumi}{5}
\item Hence prove or disprove that the following rule is admissible in $\sansbold{Mi}$:
      \begin{displaymath}
        \inferrule{\predm{\conctt{M} x x}{Mi}}{\predm{\conctt{M} x}{Mi}}{\textsf{\bf Lem}_1}
      \end{displaymath}
\item Explain why the following rule is \emph{not} admissible in $\sansbold{Mi}$.
      \begin{displaymath}
        \inferrule{\predm{x y}{Mi}}{\predm{x \conctt{IIIIII} y}{Mi}}{\textsf{\bf Lem}_2}
      \end{displaymath}
\item \textbf{A little harder:} Prove that, for all $\meta{s}$,
  $\predm{s}{Mi}$ implies
  $\predm{s}{Miu}$. Note that straightforward rule induction seems to
  require rules that you have already shown are not admissible.
  Try to use the characterisation we have already developed.
\end{enumerate}

\subsection*{2. Counting Sticks}
The following language (also presented in a similar form by Douglas Hofstadter,
but the original invention is not his) is called the $\mathbf{\Phi\Psi}$
system. Unlike the $\sansbold{Miu}$ language discussed above, this language is
not comprised of a unary judgement, but of a ternary \emph{relation}, written $x\ \mathbf{\Phi}\ y\ \mathbf{\Psi}\ z$, where $x$, $y$ and $z$ are strings of hyphens (i.e $\conctt{-}$), which may be empty ($\conc{\epsilon}$). The system is defined as follows:
\newcommand{\phipsi}[3]{\meta{#1}\ \mathbf{\Phi}\ \meta{#2}\ \mathbf{\Psi}\ \meta{#3}}
  \begin{displaymath}
\inferrule*[Left=B]{\ }{\phipsi{\conc{\epsilon}}{x}{x}} \quad \inferrule*[Right=I]{\phipsi{x}{y}{z}}{\phipsi{\conctt{-}x}{y}{\conc{\conctt{-}}z}}
  \end{displaymath}
\begin{enumerate}[label=\alph*)]
\item \textbf{Easy}: Prove that $\phipsi{\conctt{--}}{\conctt{---}}{\conctt{-----}}$.
\item Is the following rule admissible? Is it derivable? Prove your answer.
    \begin{displaymath}
      \inferrule*[Right=I']{\phipsi{\conctt{-}x}{y}{\conctt{-}z}}{\phipsi{x}{y}{z}}
    \end{displaymath}
\item Show that $\phipsi{x}{\conc{\epsilon}}{x}$, for all hyphen strings $\meta{x}$, by doing induction on the length of the hyphen string (where $\meta{x} = \conc{\epsilon}$ and $\meta{x} = \conctt{-}\meta{x'}$).
\item Show that if $\phipsi{\conctt{-}x}{y}{z}$ then $\phipsi{x}{\conctt{-}y}{z}$, for all hyphen strings $\meta{x}$, $\meta{y}$ and $\meta{z}$, by doing a rule induction on the premise.
\item Show that $\phipsi{x}{y}{z}$ implies $\phipsi{y}{x}{z}$.
\item Have you figured out what the $\mathbf{\Phi\Psi}$ system actually is? Prove that if $\phipsi{\conctt{-}^x}{\conctt{-}^y}{\conctt{-}^z}$, then $\meta{z} = \conctt{-}^{\meta{x}+\meta{y}}$ (where $\conctt{-}^{\meta{x}}$ is a hyphen string of length $\meta{x}$). 
\end{enumerate}

\subsection*{3. Ambiguity and Simultaneity}
Here is a simple grammar for a functional programming language \footnote{It's called \emph{lambda calculus}, with \emph{de Bruijn indices} syntax, not that it's relevant to the question!}:
\begin{displaymath}
  \inferrule*[Right=Var]{\meta{x} \in \mathbb{N}}
                        {\predm{x}{\sansbold{Expr}}}
                        \quad\quad\quad
  \inferrule*[Right=Appl]{\predm{e_1}{\sansbold{Expr}}\\
                          \predm{e_2}{\sansbold{Expr}}}
                         {\predm{e_1\ e_2}{\sansbold{Expr}}}
                         \quad\quad\quad
  \inferrule*[Right=Abst]{\predm{e}{\sansbold{Expr}}}
                         {\predm{\conc{\lambda} e}{\sansbold{Expr}}}
                         \quad\quad\quad
  \inferrule*[Right=Paren]{\predm{e}{\sansbold{Expr}}}
                          {\predm{\conc{(}e\conc{)}}{\sansbold{Expr}}}
  \end{displaymath}   
\begin{enumerate}[label=\alph*)]
\item Is this grammar ambiguous? If not, explain why not. If so, give an example of an expression that has multiple parse trees.
\item Develop a new (unambiguous) grammar that encodes the left associativity of application, that is $\conc{1\ 2\ 3\ 4}$ should be parsed as $\conc{((1\ 2)\ 3)\ 4}$ (modulo parentheses). Furthermore, lambda expressions should extend as far as possible, i.e $\conc{\lambda 1\ 2}$ is equivalent to $\conc{\lambda (1\ 2)}$ not $\conc{(\lambda 1)\ 2}$.
\item \textbf{Slightly more difficult}: Prove that all expressions in your grammar are representable in $\sansbold{Expr}$, that is, that your grammar describes only strings that are in $\sansbold{Expr}$.
\end{enumerate}
\newpage
\section*{Solutions}
\begin{enumerate}[label=1.\alph*)]
  \item \begin{displaymath}
        \inferrule*[Right=4]{\inferrule*[Right=5]{\inferrule*[Right=4]{\inferrule*[Right=2]{\inferrule*[Right=3]{\inferrule*[Right=3]{\inferrule*[Right=3]{\inferrule*[Right=1]{\ }{\pred{\conctt{MI}}{Miu}}}
                                     {\pred{\conctt{MII}}{Miu}}}
                                {\pred{\conctt{MIIII}}{Miu}}}
                           {\pred{\conctt{MIIIIIIII}}{Miu}}}
                      {\pred{\conctt{MIIIIIIIIU}}{Miu}}}
                 {\pred{\conctt{MIIIIIUU}}{Miu}}}{\pred{\conctt{MIIIII}}{Miu}}}
            {\pred{\conctt{MUII}}{Miu}}
      \end{displaymath}
  \item It is not derivable, but it is admissible. It is not derivable because there is no way to construct a tree that looks like this:
     
     \begin{displaymath} 
      \inferrule{\inferrule{\predm{x\conctt{IU}}{Miu}}{\vdots}}{\predm{x\conctt{I}}{Miu}}
    \end{displaymath}
      
      It is, however, \emph{admissible} because it does not change the language Miu. There is no string $\meta{x}$ that could be judged $\predm{x}{Miu}$ with this rule that could not be so judged without it. We could show this by proving the rule using rule induction.
\item We will prove that the number of $\conctt{I}$s in any string in $\sansbold{Miu}$ is not divisible by three.
      Seeing as $\conctt{MU}$ has zero $\conctt{I}$s (a multiple of three), if we prove the above, we prove that $\conctt{MU}$ is not admissible.
      
      \begin{proof}[Base case (from rule 1)]
        We see that the string $\conctt{MI}$ has only one $\conctt{I}$, which is not a multiple of three, hence we have shown our goal.
        
        \emph{Inductive case (from rule 2)}. Given that the number of $\conctt{I}$s in $\meta{x}\conctt{I}$
        is not divisible by three (our inductive hypothesis), we can easily see that the number of $\conctt{I}$s in $\meta{x}\conctt{IU}$ is identical and therefore is similarly not divisible by three.
        
        \emph{Inductive case (from rule 3)}. Let $n$ be the number of
        $\conctt{I}$s in $\conctt{M}\meta{x}$. Our inductive hypothesis is that
        $3 \nmid n$. The number of $\conctt{I}$s in $\conctt{M}\meta{x x}$,
        clearly $2n$, is similarly indivisible, i.e $3 \nmid n$ implies $3 \nmid
        2n$ (as $3$ and $2$ are co-prime).
        
        \emph{Inductive case (from rule 4)}. Let $n$ be the number of $\conctt{I}$s in $\meta{x}\conctt{III}\meta{y}$. Our inductive hypothesis is that $3 \nmid n$. The number of $\conctt{I}$s in $\meta{x} \conctt{U}\meta{y}$, clearly $n - 3$, is similarly indivisible, i.e $3 \nmid n$ implies $3 \nmid (n-3)$
        
        \emph{Inductive case (from rule 5)}. Given that the number of $\conctt{I}$s in $\meta{x}\conctt{UU}\meta{y}$ is the same as the number of $\conctt{I}$s in $\meta{x y}$, our inductive hypothesis trivially proves our goal.
        
        \medskip
        
        Thus, by induction, no string in $\sansbold{Miu}$ has a number of $\conctt{I}$s divisible by three. Therefore, $\pred{\conctt{MU}}{Miu}$ is not admissible.        
\end{proof}
      \item By rule induction on the premise:

        \begin{proof}[Base case (from rule A)] $\conctt{MI} = \conctt{M\:I}^{2^k - 6i}$ when $2^k - 6i = 1$, i.e when $k = 0$ and $i = 0$.
          
          \emph{Inductive case (from rule B)}. Given that $\conctt{M}\meta{x} =
          \conctt{MI}^{2^a - 6b}$ (our inductive hypothesis), we must show that
          $\conctt{M}\meta{x x} = \conctt{M\:I}^{2^k - 6i}$ for some $k$ and
          some $i$. As $\meta{x} = \conctt{I}^{2^a - 6b}$ (from I.H), we have
          that $\meta{x x} = \conctt{I}^{2 (2^a - 6b)} = \conctt{I}^{2^{a + 1} -
            6(2b)} = \conctt{I}^{2^k - 6i}$ for $k = a + 1$ and $i = 2b$.
          
          \emph{Inductive case (from rule C)}. Given that $\meta{x} \conctt{IIIIII} \meta{y} = \conctt{M\:I}^{2^a - 6b}$ (our inductive hypothesis). We must show that $\meta{x y} = \conctt{M\:I}^{2^k - 6i}$ for some $k$ and $i$. It should be clear to see that this rule simply subtracts six $\conctt{I}$ characters, and therefore $\meta{x y} = \conctt{MI}^{2^a - 6(b + 1)}$, hence $k = a$ and $i = b + 1$.
          
          \medskip
          
          Thus, all strings in $\sansbold{Mi}$ can be expressed as $\conctt{M\:I}^{2^k - 6i}$ where $2^k - 6i > 0$
        \end{proof}

      \item There is one structural induction for each lemma: \begin{enumerate}[label=\roman*.]
           \item By structural induction on the natural number $k$:

\begin{proof}[Base case (when $k = 0$)] We have to show $\pred{\conctt{MI}}{Mi}$, which is true by rule A.
            
            \emph{Inductive case (when $k = k' + 1$)} We have to show $\pred{\conctt{MI}^{2^{k' + 1}}}{Mi}$, with the inductive hypothesis that
             $\pred{\conctt{MI}^{2^{k'}}}{Mi}$. Equivalently, we have to show $\pred{\conctt{MI}^{2^{k'}} \conctt{I}^{2^{k'}}}{Mi}$, as follows:
             \begin{displaymath}
             \inferrule{\inferrule{\ }{\pred{\conctt{MI}^{2^{k'}}}{Mi}}{I.H}}{\pred{\conctt{MI}^{2^{k'}} \conctt{I}^{2^{k'}}}{Mi}}{B}
           \end{displaymath}
           Therefore, by induction on the natural number $k$, we have shown $\forall k. \pred{\conctt{M}\:\conctt{I}^{2^k}}{Mi}$.
          \end{proof}
          \item By induction on the natural number $i$:

\begin{proof}[Base case (when $i = 0$)] We must show that $\pred{\conctt{M}\:\conctt{I}^k}{Mi}$ implies $\pred{\conctt{M}\:\conctt{I}^{k - 0}}{Mi}$, which is obviously a tautology.
            
            \emph{Inductive case (when $i = i' + 1$)} We must show that $\pred{\conctt{M}\:\conctt{I}^k}{Mi}$ implies $\pred{\conctt{M}\:\conctt{I}^{k - 6(i' + 1)}}{Mi}$, given the inductive hypothesis that $\pred{\conctt{M}\:\conctt{I}^{k - 6i'}}{Mi}$. Note that our I.H can be restated as $\pred{\conctt{MIIIIII\:I}^{k - 6(i' + 1)}}{Mi}$ due to our assumption that $k - 6(i' + 1) > 0$. With this, we can prove our goal as shown:
            \begin{displaymath}
               \inferrule{\inferrule{\ }{\pred{\conctt{MIIIIII\:I}^{k - 6(i' + 1)}}{Mi}}{I.H}}{\pred{\conctt{M\:I}^{k - 6(i' + 1)}}{Mi}}{C}
             \end{displaymath}
          \end{proof}
          \end{enumerate}
      
      \item
        We know from the previous lemma (i) that $\predm{\conctt{M}x x}{Mi} $ implies ${\meta{x} ^ 2} = \conctt{I}^{2^k - 6i}$
         for some $k$ and some $i$ where $2^k - 6i > 0$.

        This rule is \emph{not admissible} as it adds strings to the
        language. As $2^4 - 6 = 10$, we know $\conctt{MI}^{10}$ is in the
        language. This rule would make $\conctt{MI}^5$ a string in the language.
        Without the rule, this is not a string in the language as there is no
        natural $k$ nor $i$ such that $2^k - 6i = 5$.
      \item
        The rule is not admissible as it adds strings to the language. This
        allows us to \emph{add} six $\conctt{I}$ characters to any string in
        $\sansbold{Mi}$ and judge it in $\sansbold{Mi}$, which results in
        additional strings. For example, applying the rule to $\conctt{MI}$
        (which is in Mi), gives us $\conctt{M\:I}^7$, when our
        existing formulation of Mi ($\conctt{M\:I}^{2^k - 6i}$)
        clearly only allows for even amounts of $\conctt{I}$s (or exactly one $\conctt{I}$).
       \item
      We shall show that all strings in $\sansbold{Mi}$, characterised by $\conctt{M\:I}^{2^k - 6i}$ where $2^k - 6i > 0$, are also in $\sansbold{Miu}$. That is, we shall show that $\pred{\conctt{M\:I}^{2^k - 6i}}{Miu}$.
      
      To start, we shall prove inductively on $k$ that $\pred{\conctt{M\:I}^{2^k}}{Miu}$ for all $k$.
      
      \begin{proof}[Base case (Where $k = 0$)] We must show $\pred{\conctt{MI}}{Miu}$, which we know trivially from rule 1.
      
      \emph{Inductive case (where $k = k' + 1$)}. We must show $\pred{\conctt{M\:I}^{2^{k' + 1}}}{Miu}$, given the inductive hypothesis that $\pred{\conctt{M\:I}^{2^{k'}}}{Miu}$. Note we can restate our proof goal as $\pred{\conctt{M\:I}^{2^{k'}}\conctt{I}^{2^{k'}}}{Miu}$
      
      \begin{displaymath}        \inferrule{\inferrule{\ }{\pred{\conctt{M\:I}^{2^{k'}}}{Miu}}{I.H}}{\pred{\conctt{M\:I}^{2^{k'}}\conctt{I}^{2^{k'}}}{Miu}}{B}
      \end{displaymath}
      
      \medskip
      
      Thus we have shown by induction that $\pred{\conctt{M\:I}^{2^k}}{Miu}$ for all $k$.
      
      \bigskip
      
      Next we must prove that $\pred{\conctt{M\:I}^{k}}{Miu}$ implies $\pred{\conctt{M\:I}^{k - 6i}}{Miu}$ for all $i$, assuming $k - 6i > 0$.
      
      \medskip
      
      \emph{Base case (where $i = 0$)}. We must show that  $\pred{\conctt{M\:I}^{k}}{Miu}$ implies $\pred{\conctt{M\:I}^{k - 0}}{Miu}$, which is trivially a tautology.
      
      \emph{Inductive case (where $i = i' + 1$)}. we must show that $\pred{\conctt{M\:I}^{k - 6(i' + 1)}}{Miu}$ given the inductive hypothesis $\pred{\conctt{M\:I}^{k - 6i'}}{Miu}$. As we know $k - 6(i' + 1) > 0$, we can restate our inductive hypothesis as $\pred{\conctt{MIIIIII\:I}^{k - 6(i' + 1)}}{Miu}$, and easily prove our goal:
      
      \begin{displaymath}
        \inferrule*[Right=5]{\inferrule*[Right=4]{\inferrule*[Right=4]{\inferrule*[Right=I.H]{\ }{\pred{\conctt{MIIIIII\:I}^{k - 6(i' + 1)}}{Miu}}}
            {\pred{\conctt{MUIII\:I}^{k - 6(i' + 1)}}{Miu}}}
            {\pred{\conctt{MUU\:I}^{k - 6(i' + 1)}}{Miu}}}
            {\pred{\conctt{M\:I}^{k - 6(i' + 1)}}{Miu}}
      \end{displaymath}

      Thus, as $\pred{\conctt{M\:I}^{k}}{Miu}$ implies
      $\pred{\conctt{M\:I}^{k - 6i}}{Miu}$, and
      $\pred{\conctt{M\:I}^{2^k}}{Miu}$, we can see that
      $\pred{\conctt{M\:I}^{2^k - 6i}}{Miu}$ for all $k$ and $i$
      where $2^k - 6i > 0$. As this is the exact characterisation of
      $\sansbold{Mi}$, we have proven that $\predm{s}{Mi}$ implies $\predm{s}{Miu}$ for all $\meta{s}$.
    \end{proof}
\end{enumerate}

\begin{enumerate}[label=2.\alph*)]
\item
      \begin{displaymath}
       \inferrule*[Right=I]{\inferrule*[Right=I]{\inferrule*[Right=B]{\ }{\phipsi{\conc{\epsilon}}{\conctt{---}}{\conctt{---}}}}
             {\phipsi{\conctt{-}}{\conctt{---}}{\conctt{----}}}}
             {\phipsi{\conctt{--}}{\conctt{---}}{\conctt{-----}}}
      \end{displaymath}
\item
      It is not derivable (as it cannot be shown with a proof tree), but it is
      admissible. We know this because the language definition for
      $\mathbf{\Phi\Psi}$ is unambiguous, so the only way for
      $\phipsi{\conctt{-}x}{y}{\conctt{-}z}$ to hold is if this was established
      by rule $I$. Therefore, we can deduce that $\phipsi{x}{y}{z}$ by rule inversion.
\item
        \begin{proof}[Base case (where $\meta{x} = \conc{\epsilon}$)] We must show that $\phipsi{\conc{\epsilon}}{\conc{\epsilon}}{\conc{\epsilon}}$, which is trivially true by rule $B$.

          \emph{Inductive case (where $\meta{x} = \conctt{-}\meta{x'}$)}. We
          have the inductive hypothesis $\phipsi{x'}{\conc{\epsilon}}{x'}$, and must show that $\phipsi{\conctt{-}x'}{\conc{\epsilon}}{\conctt{-}x'}$. Our goal trivially reduces to our induction hypothesis by rule $I$.
          
          Therefore we have shown $\phipsi{x}{\conc{\epsilon}}{x}$ for all
          $\meta{x}$ by induction on the length of the hyphen string $\meta{x}$.
        \end{proof}
\item By structural induction on $\meta{x}$.
        \begin{proof}[Base case (where $\phipsi{\conctt{-}}{y}{\conctt{-}y}$) ] We must show that $\phipsi{\conc{\epsilon}}{\conctt{-}y}{\conctt{-}y}$, which is trivially true by rule $B$.
          
          \emph{Inductive case (where $\phipsi{\conctt{--}x'}{y}{\conctt{-}z'}$)}. We have the inductive hypothesis:
          \begin{displaymath}
            \inferrule*[Right=I.H]{\phipsi{\conctt{-}x'}{y}{z'}}{\phipsi{x'}{\conctt{-}y}{z'}}
          \end{displaymath}
          
          We must show that $\phipsi{\conctt{-}x'}{\conctt{-}y}{\conctt{-}z'}$.
          
          \begin{displaymath}
            \inferrule*[Right=I]{\inferrule*[Right=I.H]{\inferrule*[Right=I']{\inferrule*{\ }{\phipsi{\conctt{--}x'}{y}{\conctt{-}z'}}}{\phipsi{\conctt{-}x'}{y}{z'}}}{\phipsi{x'}{\conctt{-}y}{z'}}}{\phipsi{\conctt{-}x'}{\conctt{-}y}{\conctt{-}z'}}
          \end{displaymath}
          
          Thus we have shown by induction that  if $\phipsi{\conctt{-}x}{y}{z}$ then $\phipsi{x}{\conctt{-}y}{z}$, for all hyphen strings $\meta{x}$, $\meta{y}$ and $\meta{z}$.
          
        \end{proof}
\item
        We show this by rule induction on the premise with the rules of $\mathbf{\Phi\Psi}$. 
        \begin{proof}[Base case (from rule B, where $\phipsi{\conc{\epsilon}}{y}{y}$)] We must show that $\phipsi{y}{\conc{\epsilon}}{y}$. We proved this, most fortunately, above in part (c).
          
          \medskip
          
          \emph{Inductive case (from rule I, where $\phipsi{\conctt{-}x'}{y}{{-}z'}$ )}. We have the inductive hypothesis:
          \begin{displaymath} \inferrule{\phipsi{x'}{y}{z'}}{\phipsi{y}{x'}{z'}}{I.H} \end{displaymath} 
            
            We must show that $\phipsi{y}{\conctt{-}x'}{\conctt{-}z'}$. 
          
          \begin{displaymath}            
            \inferrule*[Right=(\textup{d})]{\inferrule*[Right=I]{\inferrule*[Right=I.H]{\inferrule*[Right=I']{\inferrule*{\
                    }
                      {\phipsi{\conctt{-}x'}{y}{\conctt{-}z'}}}{\phipsi{x'}{y}{z'}}}{\phipsi{y}{x'}{z'}}}
                     {\phipsi{\conctt{-}y}{x'}{\conctt{-}z'}}
                }
                {\phipsi{y}{\conctt{-}x'}{\conctt{-}z'}}
          \end{displaymath}
          
          Thus we have shown by induction that $\phipsi{x}{y}{z}$ implies $\phipsi{y}{x}{z}$.
        \end{proof}
\item        
        We proceed by rule induction on the premise.
        
        \begin{proof}[Base case (from rule $B$, where $\phipsi{\conctt{-}^{\textcolor{black}{0}}}{\conctt{-}^y}{\conctt{-}^{y}}$)] We must show that $\phipsi{\conctt{-}^{\textcolor{black}{0}}}{\conctt{-}^y}{\conctt{-}^{\textcolor{black}{0+}y}}$. As $0 + \meta{y} = \meta{y}$, this trivially reduces to the premise.
          
          \emph{Inductive case (from rule $I$, where
            $\phipsi{\conctt{-}^{x'\textcolor{black}{+
                  1}}}{\conctt{-}^{y}}{\conctt{-}^{z'\textcolor{black}{+
                  1}}}$)}. We have the inductive hypothesis that
          $\phipsi{\conctt{-}^{x'}}{\conctt{-}^{y}}{\conctt{-}^{z'}}$ implies $\meta{z'} = \meta{x'} + \meta{y}$. We must show that $\meta{z'} + 1 = (\meta{x'} + 1) + \meta{y}$, or, equivalently, that $\meta{z'} = \meta{x'} + \meta{y}$:
          
          \begin{displaymath}
            \inferrule*[Right=I.H]{\inferrule*[Right=I]{\inferrule*{\ }{\phipsi{\conctt{-}^{x'\textcolor{black}{+ 1}}}{\conctt{-}^{y}}{\conctt{-}^{z'\textcolor{black}{+ 1}}}}}{\phipsi{\conctt{-}^{x'}}{\conctt{-}^{y}}{\conctt{-}^{z'}}}}{\meta{z'} = \meta{x'} + \meta{y}}
          \end{displaymath}
          
          Thus we have shown by rule induction that the $\mathbf{\Phi\Psi}$ system is in fact unary addition.
        \end{proof}
\end{enumerate}

\begin{enumerate}[label=3.\alph*)]
  \item
    Yes, the expression $\conc{1\ 2\ 3}$ could be parsed two different ways, i.e:
    \begin{displaymath}
      \inferrule*[Right=Appl]{\inferrule*[Left=Appl]{\inferrule*[Left=Var]{\ }{\predc{1}{Expr}}
               \\ \inferrule*[Right=Var]{\ }{\predc{2}{Expr}}
               }{\predc{1\ 2}{Expr}}\quad\inferrule*[Right=Var]{\ }{\predc{3}{Expr}}}{\predc{1\ 2\ 3}{Expr}}
    \end{displaymath}
        Or:
    \begin{displaymath}
      \inferrule*[Right=Appl]{\inferrule*[Left=Var]{\ }{\predc{1}{Expr}}\\
            \inferrule*[Right=Appl]{\inferrule*[Left=Var]{\ }{\predc{2}{Expr}}
               \\ \inferrule*[Right=Var]{\ }{\predc{3}{Expr}}
               }{\predc{2\ 3}{Expr}}
          }{\predc{1\ 2\ 3}{Expr}}
    \end{displaymath}
\item

    \begin{displaymath}
      \inferrule*[Right=AVar]{x \in \mathbb{N}}{\predm{x}{AExpr}}\quad\quad\quad\quad\inferrule*[Right=AAppl]{\predm{e_1}{PExpr}\\\predm{e_2}{AExpr}}{\predm{e_1\ e_2}{PExpr}}\quad\quad\quad\quad\inferrule*[Right=AAbs]{\predm{e}{LExpr}}{\predm{\conc{\lambda} e}{LExpr}}
      \end{displaymath}
      \begin{displaymath}
        \inferrule*[Right=AParen]{\predm{e}{LExpr}}{\predm{\conc{(}e\conc{)}}{AExpr}}\quad\quad\quad\quad
        \inferrule*[Right=Shunt$_1$]{\predm{e}{PExpr}}{\predm{e}{LExpr}}\quad\quad\quad\quad\inferrule*[Right=Shunt$_2$]{\predm{e}{AExpr}}{\predm{e}{PExpr}}   
   \end{displaymath}
   \item
    We shall prove the following simultaneously:
    \begin{itemize}
      \item $\; \predm{x}{LExpr} \implies \predm{x}{Expr}$
      \item $\; \predm{x}{PExpr} \implies \predm{x}{Expr}$
      \item $\; \predm{x}{AExpr} \implies \predm{x}{Expr}$
    \end{itemize}
    \begin{proof}[Base case (from rule $\textsc{AVar}$, where $\predm{x}{AExpr}$ for some $\meta{x} \in \mathbb{N}$)] $\;$ We must show $\predm{x}{Expr}$, trivial by rule $\textsc{Var}$
      
      \emph{Inductive case (from rule $\textsc{AAppl}$, where $\predm{e_1}{PExpr}\ (*)$, and $\predm{e_2}{AExpr}\ (**)$)}.
     We have the inductive hypotheses $\predm{e_1}{AExpr} \lor \predm{e_1}{PExpr} \lor \predm{e_1}{LExpr} \implies \predm{e_1}{Expr}\ (\textsc{I.H}_1)$ and  $\predm{e_2}{AExpr} \lor \predm{e_2}{PExpr} \lor \predm{e_2}{LExpr} \implies \predm{e_2}{Expr}\ (\textsc{I.H}_2)$. 
     
      We must show that $\predm{e_1\ e_2}{Expr}$.
      
      \begin{displaymath}
        \inferrule*[Right=Appl]{\inferrule*[Left=I.H$_1$]{\inferrule*[Left=$(*)$]{\ }{\predm{e_1}{LExpr}}}{\predm{e_1}{Expr}}\\\inferrule*[Right=I.H$_2$]{\inferrule*[Right=$(**)$]{\ }{\predm{e_2}{AExpr}}}{\predm{e_2}{Expr}}}{\predm{e_1 e_2}{Expr}}
      \end{displaymath}
      \medskip
      
      \emph{Inductive case (from rule $\textsc{AAbs}$, where
        $\predm{x}{LExpr}$)}. Applying one of our inductive hypotheses
      $\predm{x}{LExpr} \implies \predm{x}{Expr}$ to our assumption $\predm{x}{LExpr}$, we can deduce that $\predm{x}{Expr}$, and then we can apply forwards the rule \textsc{Abs} to prove our goal: $\predm{\conc{\lambda} x}{Expr}$.
      \medskip
      
      \emph{Inductive case (from rule $\textsc{AParen}$, where
        $\predm{x}{LExpr}$)}. Using one of the I.H with our assumption, we get $\predm{x}{Expr}$, then by rule \textsc{Paren} we show our goal $\predm{\conc{(}x\conc{)}}{Expr}$.
      \medskip
      
      The inductive case for the rules $\textsc{Shunt}_1$ and $\textsc{Shunt}_2$ are trivial as they do not alter the expression.
      \bigskip
      
      Thus, by induction, $\predm{s}{LExpr} \lor \predm{s}{PExpr} \lor \predm{s}{AExpr} \implies \predm{s}{Expr}$. 
      \end{proof}
\end{enumerate}
\end{document}
